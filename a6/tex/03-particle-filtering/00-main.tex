\item {\bf Particle Filtering}

Though exact inference works well for the small maps, it wastes a lot of effort
computing probabilities for {\em every available tile}, even for tiles that are
unlikely to have a car on them.  We can solve this problem using a particle
filter. Updates to the particle filter have complexity that's linear in the
number of particles, rather than linear in the number of tiles.

For a great conceptual explanation of how particle filtering works, check out
\href{https://www.youtube.com/watch?v=aUkBa1zMKv4}{this video} on using particle 
filtering to estimate an airplane's altitude. 

In this problem, you'll implement two short but important methods for the
|ParticleFilter| class in |submission.py|. When you're finished, your code
should be able to track cars nearly as effectively as it does with exact
inference.

\begin{enumerate}

  \item \points{3a}
Some of the code has been provided for you. For example, the particles have
already been initialized randomly. You need to fill in the |observe| and
|elapseTime| functions. These should modify |self.particles|, which is a map
from tiles |(row, col)| to the number of particles existing at that tile, and
|self.belief|, which needs to be updated each time you re-sample the particle locations.

You should use the same transition probabilities as in exact inference. The
belief distribution generated by a particle filter is expected to look noisier
compared to the one obtained by exact inference.
\begin{lstlisting}
python drive.py -a -i particleFilter -l lombard
\end{lstlisting}

To debug, you might want to start with the parked car flag (|-p|) and the
display car flag (|-d|).


\end{enumerate}
