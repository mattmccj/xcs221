\item \points{4a}
Write a better evaluation function for Pac-Man in the provided function
|betterEvaluationFunction|. The evaluation function should evaluate states
rather than actions. You may use any tools at your disposal for evaluation,
including any |util.py| code from the previous assignments. With depth 2 search,
your evaluation function should clear the |smallClassic| layout with two random
ghosts more than half the time for full credit and still run at a reasonable
rate.

\begin{lstlisting}
python pacman.py -l smallClassic -p ExpectimaxAgent -a evalFn=better -q -n 20
\end{lstlisting}

We will run your Pac-Man agent 20 times, and calculate the average score you
obtained in the winning games. Starting from 1300, you obtain 1 point per 100
point increase in your average winning score. In |grader.py|, you can see the
how extra credit is awarded. For example, you get 2 points if your average
winning score is between 1500 and 1600. In addition, the top 3 people in the
class will get additional points.

Check out the current top scorers on the leaderboard! You will be added
automatically when you submit. You can also access the leaderboard by opening
your submission on Gradescope and clicking "Leaderboard" on the top right
corner.

{\bf {\em Hints and Observations}}
\begin{itemize}
  \item Having gone through the rest of the assignment, you should play Pac-Man
  again yourself and think about what kinds of features you want to add to the
  evaluation function. How can you add multiple features to your evaluation
  function?
  \item You may want to use the reciprocal of important values rather than the
  values themselves for your features.
  \item The |betterEvaluationFunction| should run in the same time limit as the
  other problems.
\end{itemize}